\section{Program gto\char`_sum}
The \texttt{gto\char`_sum} adds decimal values from a file, line by line, split by spaces or tabs.\\
For help type:
\begin{lstlisting}
./gto_sum -h
\end{lstlisting}
In the following subsections, we explain the input and output parameters.

\subsection*{Input parameters}

The \texttt{gto\char`_sum} program needs two streams for the computation, namely the input and output standard. The input is a file with decimal numbers.\\

The attribution is given according to:
\begin{lstlisting}
Usage: ./gto_sum [options] [[--] args]
   or: ./gto_sum [options]

It adds decimal values in file, line by line, splitted by spaces or tabs.

    -h, --help        show this help message and exit

Basic options
    < input.num       Input numeric file (stdin)
    > output.num      Output numeric file (stdout)

Optional
    -r, --sumrows     When active, the application adds all the values line by line
    -a, --sumall      When active, the application adds all values

Example: ./gto_sum -a < input.num > output.num
\end{lstlisting}
An example of such an input file is:
\begin{lstlisting}
0.123	5	5
3.432
2.341   3   2
1.323
7.538	5
4.122
0.242 
0.654
5.633	10
\end{lstlisting}

\subsection*{Output}
The output of the \texttt{gto\char`_sum} program is the sum of the numbers of the input file.\\
Executing the application with the provided input and with the flag to add only the numbers in each row, the output of this execution is:
\begin{lstlisting}
10.123000
3.432000
7.341000
1.323000
12.538000
4.122000
0.242000
0.654000
15.633000
\end{lstlisting}