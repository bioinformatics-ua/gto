\section{Program gto\char`_amino\char`_acid\char`_to\char`_seq}
The \texttt{gto\char`_amino\char`_acid\char`_to\char`_seq} converts amino acid sequences to DNA sequences.\\
For help type:
\begin{lstlisting}
./gto_amino_acid_to_seq -h
\end{lstlisting}
In the following subsections, we explain the input and output parameters.

\subsection*{Input parameters}

The \texttt{gto\char`_amino\char`_acid\char`_to\char`_seq} program needs two streams for the computation, namely the input and output standard. The input stream is an amino acid sequence.\\
The attribution is given according to:
\begin{lstlisting}
Usage: ./gto_amino_acid_to_seq [options] [[--] args]
   or: ./gto_amino_acid_to_seq [options]

It converts amino acid sequences to DNA sequences.


    -h, --help        Show this help message and exit

Basic options
    < input.prot      Input amino acid sequence file (stdin)
    > output.seq      Output sequence file (stdout)

Example: ./gto_amino_acid_to_seq < input.prot > output.seq
\end{lstlisting}
An example of such an input file is:
\begin{lstlisting}
IPFLLKKQFALADKLVLSKLRQLLGGRIKMMPCGGAKLEPAIGLFFHAIGINIKLGYGMTETTATVSCWHDFQFNPNSI
GTLMPKAEVKIGENNEILVRGGMVMKGYYKKPEETAQAFTEDGFLKTGDAGEFDEQGNLFITDRIKELMKTSNGKYIAP
QYIESKIGKDKFIEQIAIIADAKKYVSALIVPCFDSLEEYAKQLNIKYHDRLELLKNSDILKMFE
\end{lstlisting}

\subsection*{Output}

The output of the \texttt{gto\char`_amino\char`_acid\char`_to\char`_seq} program is a DNA sequence.\\
Using the input above, an output example of this is:
\begin{lstlisting}
TO DO
\end{lstlisting}
