\section{Program gto\char`_amino\char`_acid\char`_compressor}
The \texttt{gto\char`_amino\char`_acid\char`_compressor} is a new lossless compressor to compress efficiently amino acid sequences (proteins). It uses cooperation between multiple contexts and substitutional tolerant context models. The cooperation between models is balanced with weights that benefit the models with better performance according to a forgetting function specific for each model.\\
For help type:
\begin{lstlisting}
./gto_amino_acid_compressor -h
\end{lstlisting}
In the following subsections, we explain the input and output parameters.

\subsection*{Input parameters}

The \texttt{gto\char`_amino\char`_acid\char`_compressor} program needs a file with amino acid sequences to compress.\\
The attribution is given according to:
\begin{lstlisting}
Usage: ./gto_amino_acid_compressor [OPTION]... -r [FILE]  [FILE]:[...]                          
Compression of amino acid sequences.                                   
                                                                       
Non-mandatory arguments:                                               
                                                                       
  -h                     give this help,                               
  -s                     show AC compression levels,                   
  -v                     verbose mode (more information),              
  -V                     display version number,                       
  -f                     force overwrite of output,                    
  -l <level>             level of compression [1;7] (lazy -tm setup),  
  -t <threshold>         threshold frequency to discard from alphabet,
  -e                     it creates a file with the extension ".iae" 
                         with the respective information content.      
                                                                       
  -rm <c>:<d>:<g>/<m>:<e>:<a>  reference model (-rm 1:10:0.9/0:0:0),   
  -rm <c>:<d>:<g>/<m>:<e>:<a>  reference model (-rm 5:90:0.9/1:50:0.8),
  ...                                                                  
  -tm <c>:<d>:<g>/<m>:<e>:<a>  target model (-tm 1:1:0.8/0:0:0),       
  -tm <c>:<d>:<g>/<m>:<e>:<a>  target model (-tm 7:100:0.9/2:10:0.85), 
  ...                                                                  
                         target and reference templates use <c> for    
                         context-order size, <d> for alpha (1/<d>), <g>
                         for gamma (decayment forgetting factor) [0;1),
                         <m> to the maximum sets the allowed mutations,
                         on the context without being discarded (for   
                         deep contexts), under the estimator <e>, using
                         <a> for gamma (decayment forgetting factor)   
                         [0;1) (tolerant model),                       
                                                                       
  -r <FILE>              reference file ("-rm" are loaded here),     
                                                                       
Mandatory arguments:                                                   
                                                                       
  <FILE>:<...>:<...>     file to compress (last argument). For more    
                         files use splitting ":" characters.         
                                                                       
Example:                                                               
                                                                       
  [Compress]   ./gto_amino_acid_compressor -v -tm 1:1:0.8/0:0:0 -tm 5:20:0.9/3:20:0.9 seq.txt 
  [Decompress] ./gto_amino_acid_decompressor -v seq.txt.co  
\end{lstlisting}
In the following example, nine amino acid sequences are downloaded and one of the smallest (HI) is compressed and decompressed. Finally, the uncompressed sequence is compared with the original one to verify that they are the same.
\begin{lstlisting}
wget http://sweet.ua.pt/pratas/datasets/AminoAcidsCorpus.zip
unzip AminoAcidsCorpus.zip
cp AminoAcidsCorpus/HI .
./gto_amino_acid_compressor -v -l 2 HI
./gto_amino_acid_decompressor -v HI.co
cmp HI HI.de
\end{lstlisting}