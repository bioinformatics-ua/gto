\section{Program gto\char`_fastq\char`_minimum\char`_read\char`_size}
The \texttt{gto\char`_fastq\char`_minimum\char`_read\char`_size} filters the FASTQ reads with length smaller than the value defined.\\
For help type:
\begin{lstlisting}
./gto_fastq_minimum_read_size -h
\end{lstlisting}
In the following subsections, we explain the input and output parameters.

\subsection*{Input parameters}

The \texttt{gto\char`_fastq\char`_minimum\char`_read\char`_size} program needs two streams for the computation, namely the input and output standard. The input stream is a FASTQ file.\\
The attribution is given according to:
\begin{lstlisting}
Usage: ./gto_fastq_minimum_read_size [options] [[--] args]
   or: ./gto_fastq_minimum_read_size [options]

It filters the FASTQ reads with the length smaller than the value defined. 
If present, it will erase the second header (after +).

    -h, --help            show this help message and exit

Basic options
    -s, --size=<int>      The minimum read length
    < input.fastq         Input FASTQ file format (stdin)
    > output.fastq        Output FASTQ file format (stdout)

Example: ./gto_fastq_minimum_read_size -s <size> < input.fastq > output.fastq

Console output example:
<FASTQ non-filtered reads>
Total reads    : value
Filtered reads : value
\end{lstlisting}
An example of such an input file is:
\begin{lstlisting}
@SRR001666.1 071112_SLXA-EAS1_s_7:5:1:817:345 length=60
GGGTGATGGCCGCTGCCGATGGCGTCAAATCCCACCAAGTTACCCTTAACAACTTAAGGG
+SRR001666.1 071112_SLXA-EAS1_s_7:5:1:817:345 length=60
IIIIIIIIIIIIIIIIIIIIIIIIIIIIII9IG9ICIIIIIIIIIIIIIIIIIIIIDIII
@SRR001666.2 071112_SLXA-EAS1_s_7:5:1:801:338 length=72
GTTCAGGGATACGACGTTTGTATTTTAAGAATCTGAAGCAGAAGTCGATGATAATACGCGTCGTTTTATCAT
+SRR001666.2 071112_SLXA-EAS1_s_7:5:1:801:338 length=72
IIIIIIIIIIIIIIIIIIIIIIIIIIIIIIII6IBIIIIIIIIIIIIIIIIIIIIIIIGII>IIIII-I)8I
\end{lstlisting}

\subsection*{Output}
The output of the \texttt{gto\char`_fastq\char`_minimum\char`_read\char`_size} program is a set of all filtered FASTQ reads, followed by the execution report.
The execution report only appears in the console.\\
Using the input above with 65 as size value, an output example of this is:
\begin{lstlisting}
@SRR001666.2 071112_SLXA-EAS1_s_7:5:1:801:338 length=72
GTTCAGGGATACGACGTTTGTATTTTAAGAATCTGAAGCAGAAGTCGATGATAATACGCGTCGTTTTATCAT
+
IIIIIIIIIIIIIIIIIIIIIIIIIIIIIIII6IBIIIIIIIIIIIIIIIIIIIIIIIGII>IIIII-I)8I
Total reads    : 2
Filtered reads : 1
\end{lstlisting}
