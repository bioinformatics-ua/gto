\section{Program gto\char`_fastq\char`_mutate}
The \texttt{gto\char`_fastq\char`_mutate} creates a synthetic mutation of a FASTQ file given specific rates of mutations, deletions and additions. All these parameters are defined by the user, and they are optional.\\
For help type:
\begin{lstlisting}
./gto_fastq_mutate -h
\end{lstlisting}
In the following subsections, we explain the input and output parameters.

\subsection*{Input parameters}

The \texttt{gto\char`_fastq\char`_mutate} program needs two streams for the computation, namely the input and output standard. However, optional settings can be supplied too, such as the starting point to the random generator, and the edition, deletion and insertion rates. The user can also choose to use the ACGTN alphabet in the synthetic mutation. The input stream is a FASTQ File.\\
The attribution is given according to:
\begin{lstlisting}
Usage: ./gto_fastq_mutate [options] [[--] args]
   or: ./gto_fastq_mutate [options]

Creates a synthetic mutation of a FASTQ file given specific rates of mutations,
deletions and additions

    -h, --help                    show this help message and exit

Basic options
    < input.fasta                 Input FASTQ file format (stdin)
    > output.fasta                Output FASTQ file format (stdout)

Optional
    -s, --seed=<int>              Starting point to the random generator
    -m, --mutation-rate=<dbl>     Defines the mutation rate (default 0.0)
    -d, --deletion-rate=<dbl>     Defines the deletion rate (default 0.0)
    -i, --insertion-rate=<dbl>    Defines the insertion rate (default 0.0)
    -a, --ACGTN-alphabet          When active, the application uses the ACGTN alphabet

Example: ./gto_fastq_mutate -s <seed> -m <mutation rate> -d <deletion rate> -i 
<insertion rate> -a < input.fastq > output.fastq

\end{lstlisting}
An example of such an input file is:
\begin{lstlisting}
@SRR001666.1 071112_SLXA-EAS1_s_7:5:1:817:345 length=72
GGGTGATGGCCGCTGCCGATGGCGTCAAATCCCACCAAGTTACCCTTAACAACTTAAGGGTTTTCAAATAGA
+SRR001666.1 071112_SLXA-EAS1_s_7:5:1:817:345 length=72
IIIIIIIIIIIIIIIIIIIIIIIIIIIIII9IG9ICIIIIIIIIIIIIIIIIIIIIDIIIIIII>IIIIII/
@SRR001666.2 071112_SLXA-EAS1_s_7:5:1:801:338 length=72
GTTCAGGGATACGACGTTTGTATTTTAAGAATCTGAAGCAGAAGTCGATGATAATACGCGTCGTTTTATCAT
+SRR001666.2 071112_SLXA-EAS1_s_7:5:1:801:338 length=72
IIIIIIIIIIIIIIIIIIIIIIIIIIIIIIII6IBIIIIIIIIIIIIIIIIIIIIIIIGII>IIIII-I)8I
\end{lstlisting}

\subsection*{Output}
The output of the \texttt{gto\char`_fastq\char`_mutate} program is a FASTQ file with the synthetic mutation of the input file.\\
Using the input above with 1 as seed value and value 0.5 as mutation rate, an output example of this is:
\begin{lstlisting}
@SRR001666.1 071112_SLXA-EAS1_s_7:5:1:817:345 length=72
GGACTTTGAGGTGTGGCGATAGACTGAAAACACTTCAGGGTAAAATCACTCGCAAAAGTGCTATGGTTATGG
+SRR001666.1 071112_SLXA-EAS1_s_7:5:1:817:345 length=72
IIIIIIIIIIIIIIIIIIIIIIIIIIIIII9IG9ICIIIIIIIIIIIIIIIIIIIIDIIIIIII>IIIIII/
@SRR001666.2 071112_SLXA-EAS1_s_7:5:1:801:338 length=72
GTTCAGAGCCTTTACCGTAGGGGTGTAAGATTTTATACAAAAAGTCCAGGTCAAGAGGAATCGGACAACCGA
+SRR001666.2 071112_SLXA-EAS1_s_7:5:1:801:338 length=72
IIIIIIIIIIIIIIIIIIIIIIIIIIIIIIII6IBIIIIIIIIIIIIIIIIIIIIIIIGII>IIIII-I)8I
\end{lstlisting}
