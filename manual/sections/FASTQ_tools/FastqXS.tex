\section{Program gto\char`_fastq\char`_xs}
The \texttt{gto\char`_fastq\char`_xs} is a skilled FASTQ read simulation tool, flexible, portable (does not need a reference sequence) and tunable in terms of sequence complexity. XS handles Ion Torrent, Roche-454, Illumina and ABI-SOLiD simulation sequencing types. It has several running modes, depending on the time and memory available, and aims to test computing infrastructures, namely cloud computing of large-scale projects, and test FASTQ compression algorithms. Moreover, XS offers the possibility of simulating the three main FASTQ components individually (headers, DNA sequences and quality-scores). Quality-scores can be simulated using uniform and Gaussian distributions.\\
For help type:
\begin{lstlisting}
./gto_fastq_xs -h
\end{lstlisting}
In the following subsections, we explain the input and output parameters.

\subsection*{Input parameters}

The \texttt{gto\char`_fastq\char`_xs} program needs a FASTQ file to compute.\\
The attribution is given according to:
\begin{lstlisting}
Usage: ./gto_fastq_xs   [OPTION]... [FILE] 

System options:
 -h                       give this help
 -v                       verbose mode

Main FASTQ options:
 -t  <sequencingType>     type: 1=Roche-454, 2=Illumina, 3=ABI SOLiD, 4=Ion Torrent
 -hf <headerFormat>       header format: 1=Length appendix, 2=Pair End
 -i  n=<instrumentName>   the unique instrument name (use n= before name)
 -o                       use the same header in third line of the read
 -ls <lineSize>           static line (bases/quality scores) size
 -ld <minSize>:<maxSize>  dynamic line (bases/quality scores) size
 -n  <numberOfReads>      number of reads per file

DNA options:
 -f  <A>,<C>,<G>,<T>,<N>  symbols frequency
 -rn <numberOfRepeats>    repeats: number (default: 0)
 -ri <repeatsMinSize>     repeats: minimum size
 -ra <repeatsMaxSize>     repeats: maximum size
 -rm <mutationRate>       repeats: mutation frequency
 -rr                      repeats: use reverse complement repeats

Quality scores options:
 -qt <assignmentType>     quality scores distribution: 1=uniform, 2=gaussian
 -qf <statsFile>          load file: mean, standard deviation (when: -qt 2)
 -qc <template>           custom template ascii alphabet

Filtering options:
 -eh                      excludes the use of headers from output
 -eo                      excludes the use of optional headers (+) from output
 -ed                      excludes the use of DNA bases from output
 -edb                     excludes '\n' when DNA bases line size is reached
 -es                      excludes the use of quality scores from output

Stochastic options:
 -s  <seed>               generation seed

<genFile>                 simulated output file

Common usage:
 ./XS -v -t 1 -i n=MySeq -ld 30:80 -n 20000 -qt=1 -qc 33,36,39:43 File
 ./XS -v -ls 100 -n 10000 -eh -eo -es -edb -f 0.3,0.2,0.2,0.3,0.0 -rn 50 -ri 300 -ra 3000 -rm 0.1 File
\end{lstlisting}
An example of such an input file is:
\begin{lstlisting}
@SRR001666.1 071112_SLXA-EAS1_s_7:5:1:817:345 length=60
GGGTGATGGCCGCTGCCGATGGCGTCAAATCCCACCAAGTTACCCTTAACAACTTAAGGG
+
IIIIIIIIIIIIIIIIIIIIIIIIIIIIII9IG9ICIIIIIIIIIIIIIIIIIIIIDIII
@SRR001666.2 071112_SLXA-EAS1_s_7:5:1:801:338 length=72
GTTCAGGGATACGACGTTTGTATTTTAAGAATCTGAAGCAGAAGTCGATGATAATACGCGTCGTTTTATCAT
+
IIIIIIIIIIIIIIIIIIIIIIIIIIIIIIII6IBIIIIIIIIIIIIIIIIIIIIIIIGII>IIIII-I)8I
\end{lstlisting}

\subsection*{Output}
The output of the \texttt{gto\char`_fastq\char`_xs} program is a FASTQ file\\
Using the input above and the common usage with 5 reads (-n 5), an output example of this is:
\begin{lstlisting}
@output.fastq.598 LQGQLWH01D5WVZ length=62
TTCNTNCCAGGTAAAGAGAACATNCCGNCGCACTACTCGTAAGACTTGCTGGNCGAGAAAGG
+
)(+!*!$')($(()+'))$$()'!)!$!!$*+)+''('!)))!+!)(!+!*$!'$*)**++!
@output.fastq.1510 LQGQLWH01A7LJI length=57
CTAGACTACTCGAGCACTAGGCTCGCGTNTACCANGGGGNCTGCGNGTTGGCNCGGT
+
)+(*(+$*)+!*)!'!!(!(!!(*'$!+!(()$'!!+*+!!))!*!')***+!$+''
@output.fastq.2153 LQGQLWH01CHBQJ length=33
ACTTTTTGCTCAAGCAGGGTTGCCTAGCAANAC
+
*)++!+$''')*)**!+)$(*((*)$!'!+!!*
@output.fastq.3251 LQGQLWH01C8OY4 length=75
TCTTTCCTTCNCGNCCNAATTCCCCATAANAACTTAAAATCNCNNGCTGCGCGTGATCAACAATATTAATACTCC
+
!*''+*'!''!+!!!*!'!+(++)*(*($!!*((')$*!$(!'!!'+)$+*!$*!**!'()$!*'+'*'+!!+'(
@output.fastq.3934 LQGQLWH01AQDXM length=36
GGTAACNNGGAATTCTTCCAATTANCCNTGTCCGGC
+
$+)'!'!!)+)+!''**$$*!!')!+)!)*()!))$
\end{lstlisting}