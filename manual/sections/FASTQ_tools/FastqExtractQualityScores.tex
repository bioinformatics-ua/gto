\section{Program gto\char`_fastq\char`_extract\char`_quality\char`_scores}
The \texttt{gto\char`_fastq\char`_extract\char`_quality\char`_scores} extracts all quality-scores from FASTQ reads.\\
For help type:
\begin{lstlisting}
./gto_fastq_extract_quality_scores -h
\end{lstlisting}
In the following subsections, we explain the input and output paramters.

\subsection*{Input parameters}

The \texttt{gto\char`_fastq\char`_extract\char`_quality\char`_scores} program needs two streams for the computation, namely the input and output standard. The input stream is a FASTQ file.\\
The attribution is given according to:
\begin{lstlisting}
Usage: ./gto_fastq_extract_quality_scores [options] [[--] args]
   or: ./gto_fastq_extract_quality_scores [options]

It extracts all the quality-scores from FASTQ reads.

    -h, --help            show this help message and exit

Basic options
    < input.fastq         Input FASTQ file format (stdin)
    > output.fastq        Output FASTQ file format (stdout)

Example: ./gto_fastq_extract_quality_scores < input.fastq > output.fastq

Console output example:
<FASTQ quality scores>
Total reads          : value
Total Quality-Scores : value
\end{lstlisting}
An example of such an input file is:
\begin{lstlisting}
@SRR001666.1 071112_SLXA-EAS1_s_7:5:1:817:345 length=72
GGGTGATGGCCGCTGCCGATGGCGTCAAATCCCACCAAGTTACCCTTAACAACTTAAGGGTTTTCAAATAGA
+SRR001666.1 071112_SLXA-EAS1_s_7:5:1:817:345 length=72
IIIIIIIIIIIIIIIIIIIIIIIIIIIIII9IG9ICIIIIIIIIIIIIIIIIIIIIDIIIIIII>IIIIII/
@SRR001666.2 071112_SLXA-EAS1_s_7:5:1:801:338 length=72
GTTCAGGGATACGACGTTTGTATTTTAAGAATCTGAAGCAGAAGTCGATGATAATACGCGTCGTTTTATCAT
+SRR001666.2 071112_SLXA-EAS1_s_7:5:1:801:338 length=72
IIIIIIIIIIIIIIIIIIIIIIIIIIIIIIII6IBIIIIIIIIIIIIIIIIIIIIIIIGII>IIIII-I)8I
\end{lstlisting}

\subsection*{Output}
The output of the \texttt{gto\char`_fastq\char`_extract\char`_quality\char`_scores} program is a set of all quality scores from FASTQ reads, followed by the execution report.
The execution report only appears in the console.\\
Using the input above, an output example of this is the following:
\begin{lstlisting}
IIIIIIIIIIIIIIIIIIIIIIIIIIIIII9IG9ICIIIIIIIIIIIIIIIIIIIIDIIIIIII>IIIIII/
IIIIIIIIIIIIIIIIIIIIIIIIIIIIIIII6IBIIIIIIIIIIIIIIIIIIIIIIIGII>IIIII-I)8I
Total reads          : 2
Total Quality-Scores : 144
\end{lstlisting}
