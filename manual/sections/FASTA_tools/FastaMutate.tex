\section{Program gto\char`_fasta\char`_mutate}
The \texttt{gto\char`_fasta\char`_mutate} creates a synthetic mutation of a FASTA file given specific rates of editions, deletions and additions. All these parameters are defined by the user, and are optional.\\
For help type:
\begin{lstlisting}
./gto_fasta_mutate -h
\end{lstlisting}
In the following subsections, we explain the input and output parameters.

\subsection*{Input parameters}

The \texttt{gto\char`_fasta\char`_mutate} program needs two streams for the computation, namely the input and output standard. However, optional settings can be supplied too, such as the starting point to the random generator, and the edition, deletion and insertion rates. The user can also choose to use the ACGTN alphabet in the synthetic mutation. The input stream is a FASTA or Multi-FASTA file.\\
The attribution is given according to:
\begin{lstlisting}
Usage: ./gto_fasta_mutate [options] [[--] args]
   or: ./gto_fasta_mutate [options]

Creates a synthetic mutation of a fasta file given specific rates of editions, 
deletions and additions

    -h, --help                    show this help message and exit

Basic options
    < input.fasta                 Input FASTA or Multi-FASTA file format (stdin)
    > output.fasta                Output FASTA or Multi-FASTA file format (stdout)

Optional
    -s, --seed=<int>              Starting point to the random generator
    -e, --edit-rate=<dbl>         Defines the edition rate (default 0.0)
    -d, --deletion-rate=<dbl>     Defines the deletion rate (default 0.0)
    -i, --insertion-rate=<dbl>    Defines the insertion rate (default 0.0)
    -a, --ACGTN-alphabet          When active, the application uses the ACGTN alphabet

Example: ./gto_fasta_mutate -s <seed> -e <edit rate> -d <deletion rate> -i 
<insertion rate> -a < input.mfasta > output.fasta
\end{lstlisting}
An example of such an input file is:
\begin{lstlisting}
>AB000264 |acc=AB000264|descr=Homo sapiens mRNA 
ACAAGACGGCCTCCTGCTGCTGCTGCTCTCCGGGGCCACGGCCCTGGAGGGTCCACCGCTGCCCTGCTGCCATTGTCCCC
GGCCCCACCTAAGGAAAAGCAGCCTCCTGACTTTCCTCGCTTGGGCCGAGACAGCGAGCATATGCAGGAAGCGGCAGGAA
GTGGTTTGAGTGGACCTCCGGGCCCCTCATAGGAGAGGAAGCTCGGGAGGTGGCCAGGCGGCAGGAAGCAGGCCAGTGCC
GCGAATCCGCGCGCCGGGACAGAATCTCCTGCAAAGCCCTGCAGGAACTTCTTCTGGAAGACCTTCTCCACCCCCCCAGC
TAAAACCTCACCCATGAATGCTCACGCAAGTTTAATTACAGACCTGAA
>AB000263 |acc=AB000263|descr=Homo sapiens mRNA 
ACAAGATGCCATTGTCCCCCGGCCTCCTGCTGCTGCTGCTCTCCGGGGCCACGGCCACCGCTGCCCTGCCCCTGGAGGGT
GGCCCCACCGGCCGAGACAGCGAGCATATGCAGGAAGCGGCAGGAATAAGGAAAAGCAGCCTCCTGACTTTCCTCGCTTG
GTGGTTTGAGTGGACCTCCCAGGCCAGTGCCGGGCCCCTCATAGGAGAGGAAGCTCGGGAGGTGGCCAGGCGGCAGGAAG
GCGCACCCCCCCAGCAATCCGCGCGCCGGGACAGAATGCCCTGCAGGAACTTCTTCTGGAAGACCTTCTCCTCCTGCAAA
TAAAACCTCACCCATGAATGCTCACGCAAGTTTAATTACAGACCTGAA
\end{lstlisting}

\subsection*{Output}
The output of the \texttt{gto\char`_fasta\char`_mutate} program is a FASTA or Multi-FASTA file with the synthetic mutation of the input file.\\
Using the input above with 1 as seed value and value 0.5 as edition rate, an output example of this is:
\begin{lstlisting}
>AB000264 |acc=AB000264|descr=Homo sapiens mRNA 
ACGCAACGNATTCCTGCTGATCATANTGTNCCGCNCCCCNGCGACGGGGNCTCNCNNGCACACATNGTACCATTGTCCAC
NCTTNCANGTNANCGCTAGCAGGCTACNGTTTNTCCTCNCCTANNCCAANCNGGCGTNNNTACACTGGCACGTGCAGGCA
TNGGTCGGCNGGNNCCTCCGGNAACGGCACCGGAGACGAAGCTCGGNGGNTATACAGGTGTCANGAAACATCCCCGCGNC
GNGTGNCCNNGAANCCANAGAGTATCTCACTCACAACCCTGCGTGCACNTCTAGAGNANGACCTTACNCACCNTCCCNTT
NNGTACCACACCAATGAACGCTGCAGAAAGTCTGTTTNNAGGNGNGCA
>AB000263 |acc=AB000263|descr=Homo sapiens mRNA 
ATTTGAAGGCAANCGGNCCAGNAATNCGGNGGGTGCNGCTCNTGTNGGCTACGGNCATCGCGGCCCTGCTNTANTAAGCN
TGAACCACCGNTCGNNGCACTTAGCAATNGCGNAANCCGTCGGCACGGCGGAGACNAANCCGCTANTNNTTTCCCGCTNA
ATGGNTGTACAAGACCNACTANACCANCCTCCGTCACCACACTGGAGCGCANGATGGNNCGCTGNCTAGNAGNCNNTGAG
GCGCTCCNTCCTANAAANCCGTGGNCGAGCNCCCTATGGNAGNGTGGGGGTTTTACCGGAAGACCNTCGNGCCCTATGGG
AGCAATCANAANCTAGAAAGCTTACNGATGGTGANGAANTAGACTANG
\end{lstlisting}